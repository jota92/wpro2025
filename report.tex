%%%%%%%%%%%%%%%%%%%%%%%%%%%%%%%%%%%%%%%%%%%%%%%%%%%%%%%%%%%%
% wpro2025.tex
% 「基本的なデータ表示のWebアプリ開発」レポート課題 仕様書
%%%%%%%%%%%%%%%%%%%%%%%%%%%%%%%%%%%%%%%%%%%%%%%%%%%%%%%%%%%%
\documentclass[uplatex,dvipdfmx]{jlreq}

% 日本語+リンク
\usepackage[dvipdfmx]{hyperref}
\usepackage{pxjahyper}

% 表
\usepackage{booktabs}

%%%%%%%%%%%%%%%%%%%%%%%%%%%%%%%%%%%%%%%%%%%%%%%%%%%%%%%%%%%%
\title{基本的なデータ表示のWebアプリ開発 仕様書}
\author{25G1XXX~~氏名} % ←学籍番号・氏名に書き換え
\date{\today}
%%%%%%%%%%%%%%%%%%%%%%%%%%%%%%%%%%%%%%%%%%%%%%%%%%%%%%%%%%%%

\begin{document}
\maketitle

%%%%%%%%%%%%%%%%%%%%%%%%%%%%%%%%%%%%%%%%%%%%%%%%%%%%%%%%%%%%
\section*{GitHub リポジトリ URL}

本レポートで説明する 3 つの Web アプリケーションのソースコード一式は,
次の GitHub リポジトリで管理している.

\bigskip
\noindent
\url{https://github.com/jota92/wpro2025} 
\bigskip

リポジトリ内には,以下の 3 つのディレクトリを用意し,
それぞれ独立した Web アプリケーションとして実装した.

\begin{itemize}
  \item \texttt{tasks\_app}:授業課題・テスト管理アプリ
  \item \texttt{subs\_app}:サブスク管理アプリ
  \item \texttt{todos\_app}:ToDo リストアプリ
\end{itemize}

いずれのアプリケーションも,授業で説明されたとおり,
\texttt{Node.js},\texttt{Express},\texttt{EJS} を用いて実装し,
各 \texttt{index.js} の先頭には \verb|"use strict";| を付与している.
授業で説明していない外部ライブラリやフレームワークは使用していない.

%%%%%%%%%%%%%%%%%%%%%%%%%%%%%%%%%%%%%%%%%%%%%%%%%%%%%%%%%%%%
\section{利用者向け仕様}

この節では,実際に Web アプリケーションを利用するユーザの視点から,
3 つのアプリケーションの目的と画面構成,および主な操作を説明する.

\subsection{授業課題・テスト管理アプリ(\texttt{tasks\_app})}

\subsubsection*{目的}

大学の授業ごとに課題やテストが複数存在するため,
「どの授業で,いつまでに,どの課題やテストがあるか」を一覧で把握できるようにすることを目的とする.
本アプリケーションでは,授業名,課題名,種別,締切日,状態を一元管理し,
締切の近い課題を見落とさないようにする.

\subsubsection*{画面構成}

\begin{itemize}
  \item \textbf{一覧画面(トップ,\texttt{/})}  
        登録済みの課題とテストを表形式で一覧表示する.
        各行には「タイトル」「科目」「種別」「締切日」「状態」が表示され,
        「詳細」リンクから詳細画面へ遷移できる.
        画面上部には「新しい課題・テストを登録」リンクがあり,
        新規登録フォームへ移動する.

  \item \textbf{詳細画面(\texttt{/tasks/:id})}  
        選択した 1 件の課題・テストについて,
        ID,タイトル,科目,種別,締切日,状態,メモを表示する.
        「編集」リンクから編集フォームに遷移し,
        「削除」ボタンを押すと削除処理を行う.

  \item \textbf{新規登録画面(\texttt{/create})}  
        タイトル,科目,種別,締切日,状態,メモを入力し,
        新しい課題・テストを登録するフォームである.
        送信後は一覧画面へリダイレクトされ,
        登録内容を確認できる.

  \item \textbf{編集画面(\texttt{/edit/:id})}  
        既存の課題・テストの内容を変更するフォームである.
        送信後は当該課題の詳細画面へ遷移し,変更結果を確認できる.
\end{itemize}

\subsubsection*{主な操作}

利用者は次の操作を行う.

\begin{itemize}
  \item 登録済みの課題・テストの一覧を確認する.
  \item 新しい課題・テストを登録する.
  \item 進捗に応じて状態(未提出/勉強中/提出済)を更新する.
  \item 詳細画面でメモを追記し,提出方法や注意事項を残す.
  \item 不要になった課題・テストを削除する.
\end{itemize}

%%%%%%%%%%%%%%%%%%%%%%%%%%%%%%%%%%%%%%%%%%%%%%%%%%%%%%%%%%%%
\subsection{サブスク管理アプリ(\texttt{subs\_app})}

\subsubsection*{目的}

動画配信サービスや音楽配信サービスなどのサブスクリプションが増えると,
どのサービスにいくら支払っているか把握しづらくなる.
本アプリケーションは,契約中のサブスクを一覧管理し,
サービス名,カテゴリ,料金,課金サイクルなどを整理して確認できることを目的とする.

\subsubsection*{画面構成}

\begin{itemize}
  \item \textbf{一覧画面(トップ,\texttt{/})}  
        登録済みサブスクを表形式で一覧表示する.
        各行には「サービス名」「カテゴリ」「料金」「課金サイクル」が表示され,
        「詳細」リンクから詳細画面へ遷移できる.
        画面上部のリンクから新規登録フォームへ移動する.

  \item \textbf{詳細画面(\texttt{/subs/:id})}  
        1 件のサブスクについて,ID,サービス名,カテゴリ,
        料金,課金サイクル,メモを表示する.
        「編集」リンクと「削除」ボタンを備える.

  \item \textbf{新規登録画面(\texttt{/create})}  
        サービス名,カテゴリ,月額料金,課金サイクル,メモを入力して登録するフォームである.

  \item \textbf{編集画面(\texttt{/edit/:id})}  
        既存のサブスク情報を変更するフォームであり,
        料金変更やメモの追記などを行う.
\end{itemize}

\subsubsection*{主な操作}

\begin{itemize}
  \item 現在契約しているサブスク一覧を確認する.
  \item 新たに契約したサブスクを登録する.
  \item 料金変更や契約状況の変化に応じて内容を編集する.
  \item 解約したサブスクを削除して一覧を整理する.
\end{itemize}

%%%%%%%%%%%%%%%%%%%%%%%%%%%%%%%%%%%%%%%%%%%%%%%%%%%%%%%%%%%%
\subsection{ToDo リストアプリ(\texttt{todos\_app})}

\subsubsection*{目的}

日々の作業ややるべきことを簡単に整理できる ToDo リストを提供する.
タスクのタイトル,優先度,完了状況などを一覧で管理し,
重要なタスクを見落とさないようにすることを目的とする.

\subsubsection*{画面構成}

\begin{itemize}
  \item \textbf{一覧画面(トップ,\texttt{/})}  
        登録済みの ToDo を表形式で一覧表示する.
        各行には「タイトル」「優先度」「完了フラグ」が表示され,
        「詳細」リンクから詳細画面へ遷移できる.
        画面上部のリンクから新規登録フォームへ移動する.

  \item \textbf{詳細画面(\texttt{/todos/:id})}  
        1 件の ToDo について,ID,タイトル,優先度,完了フラグ,メモを表示する.
        編集と削除の操作を行うことができる.

  \item \textbf{新規登録画面(\texttt{/create})}  
        タイトル,優先度,完了フラグ(チェックボックス),メモを入力して
        新しい ToDo を登録するフォームである.

  \item \textbf{編集画面(\texttt{/edit/:id})}  
        既存の ToDo の内容を変更するフォームであり,
        タイトルの修正や完了状態の変更を行う.
\end{itemize}

\subsubsection*{主な操作}

\begin{itemize}
  \item 登録済み ToDo の一覧を確認する.
  \item 新しい ToDo を追加する.
  \item 進捗に応じて優先度や完了状態を更新する.
  \item 完了済みまたは不要な ToDo を削除する.
\end{itemize}

%%%%%%%%%%%%%%%%%%%%%%%%%%%%%%%%%%%%%%%%%%%%%%%%%%%%%%%%%%%%
\section{管理者向け仕様}

この節では,データの構造や入力ルールなど,
運用・データ管理の観点から 3 つのアプリケーションを説明する.
いずれのアプリケーションも,
データはサーバプログラム内の配列変数に保持し,
データベース管理システムは用いていない.
サーバを再起動すると配列は初期値に戻る.

%%%%%%%%%%%%%%%%%%%%%%%%%%%%%%%%%%%%%%%%%%%%%%%%%%%%%%%%%%%%
\subsection{授業課題・テスト管理アプリのデータ仕様}

課題・テストは,表\ref{tab:task-fields} に示す項目を持つオブジェクトとして管理する.

\begin{table}[h]
  \centering
  \caption{課題・テストのデータ項目}
  \label{tab:task-fields}
  \begin{tabular}{lll}
    \toprule
    項目名 & 型 & 説明 \\
    \midrule
    \texttt{id}      & 数値      & 課題・テストを一意に識別する ID.1 以上の整数. \\
    \texttt{title}   & 文字列    & 課題またはテストのタイトル. \\
    \texttt{course}  & 文字列    & 科目名(例:Webプログラミング). \\
    \texttt{type}    & 文字列    & 種別(レポート,テスト など). \\
    \texttt{dueDate} & 文字列    & 締切日(例:\texttt{2025-01-31}). \\
    \texttt{status}  & 文字列    & 状態(未提出/勉強中/提出済). \\
    \texttt{memo}    & 文字列    & 任意のメモ.空文字列も可. \\
    \bottomrule
  \end{tabular}
\end{table}

入力および運用上のルールは以下のとおりである.

\begin{itemize}
  \item \texttt{title},\texttt{course},\texttt{type},\texttt{dueDate},\texttt{status} は必須項目とする.
  \item \texttt{status} は「未提出」「勉強中」「提出済」のいずれかを選択する.
  \item \texttt{dueDate} は HTML の日付入力欄を用い,\texttt{YYYY-MM-DD} 形式の文字列として扱う.
  \item 削除操作を行った課題は配列から完全に削除される.元のデータを復元する機能は持たない.
\end{itemize}

%%%%%%%%%%%%%%%%%%%%%%%%%%%%%%%%%%%%%%%%%%%%%%%%%%%%%%%%%%%%
\subsection{サブスク管理アプリのデータ仕様}

サブスクリプション情報は,表\ref{tab:subs-fields} に示す項目を持つオブジェクトとして管理する.

\begin{table}[h]
  \centering
  \caption{サブスクのデータ項目}
  \label{tab:subs-fields}
  \begin{tabular}{lll}
    \toprule
    項目名 & 型 & 説明 \\
    \midrule
    \texttt{id}       & 数値      & サブスクを一意に識別する ID. \\
    \texttt{name}     & 文字列    & サービス名(例:Netflix). \\
    \texttt{category} & 文字列    & カテゴリ(動画配信/音楽/クラウドストレージ など). \\
    \texttt{fee}      & 数値      & 月額料金(円).0 以上の整数を想定. \\
    \texttt{cycle}    & 文字列    & 課金サイクル(毎月/毎年 など). \\
    \texttt{memo}     & 文字列    & 備考・利用目的など. \\
    \bottomrule
  \end{tabular}
\end{table}

運用ルールは以下のとおりである.

\begin{itemize}
  \item \texttt{name},\texttt{category},\texttt{fee},\texttt{cycle} は必須項目とする.
  \item \texttt{fee} は負の値を許可せず,フォーム上では 0 以上の整数のみ入力できるようにしている.
  \item サブスクを解約した場合は,削除処理で一覧から取り除く運用を想定する.
\end{itemize}

%%%%%%%%%%%%%%%%%%%%%%%%%%%%%%%%%%%%%%%%%%%%%%%%%%%%%%%%%%%%
\subsection{ToDo リストアプリのデータ仕様}

ToDo データは,表\ref{tab:todo-fields} に示す項目を持つオブジェクトとして管理する.

\begin{table}[h]
  \centering
  \caption{ToDo のデータ項目}
  \label{tab:todo-fields}
  \begin{tabular}{lll}
    \toprule
    項目名 & 型 & 説明 \\
    \midrule
    \texttt{id}       & 数値      & ToDo を一意に識別する ID. \\
    \texttt{title}    & 文字列    & ToDo の内容を表すタイトル. \\
    \texttt{priority} & 文字列    & 優先度(高/中/低). \\
    \texttt{done}     & 真偽値    & 完了フラグ(\texttt{true}:完了,\texttt{false}:未完了). \\
    \texttt{memo}     & 文字列    & 補足メモ. \\
    \bottomrule
  \end{tabular}
\end{table}

運用ルールは以下のとおりである.

\begin{itemize}
  \item \texttt{title} は必須項目とし,空のまま登録することは想定しない.
  \item \texttt{priority} は「高」「中」「低」の中から選択させる.
  \item \texttt{done} はチェックボックスで入力し,チェック有無を真偽値に変換して保持する.
\end{itemize}

%%%%%%%%%%%%%%%%%%%%%%%%%%%%%%%%%%%%%%%%%%%%%%%%%%%%%%%%%%%%
\section{開発者向け仕様}

この節では,実装者向けに技術的な仕様をまとめる.
3 つのアプリケーションはいずれも同様の構成とし,
授業で扱った範囲の技術のみを用いて実装した.

%%%%%%%%%%%%%%%%%%%%%%%%%%%%%%%%%%%%%%%%%%%%%%%%%%%%%%%%%%%%
\subsection{共通の実装方針}

\begin{itemize}
  \item サーバサイドは \texttt{Node.js} と \texttt{Express} を用いる.
  \item テンプレートエンジンとして \texttt{EJS} を利用し,
        \texttt{views} ディレクトリ以下にテンプレートファイルを配置する.
  \item 静的ファイル(CSS)は \texttt{public} ディレクトリに配置し,
        \verb|app.use("/public", express.static(...))| で配信する.
  \item フォームから送信されるデータの取得には
        \verb|express.urlencoded({ extended: true })| を用いる.
  \item 各アプリケーションの \texttt{index.js} の先頭に
        \verb|"use strict";| を記述する.
  \item データはサーバプログラム内の配列変数(\texttt{tasks},\texttt{subscriptions},
        \texttt{todos})として保持し,授業で扱っていないデータベースは使用しない.
\end{itemize}

%%%%%%%%%%%%%%%%%%%%%%%%%%%%%%%%%%%%%%%%%%%%%%%%%%%%%%%%%%%%
\subsection{授業課題・テスト管理アプリのエンドポイント}

授業課題・テスト管理アプリ(\texttt{tasks\_app})の主なエンドポイントを表\ref{tab:task-api} に示す.

\begin{table}[h]
  \centering
  \caption{授業課題・テスト管理アプリのエンドポイント}
  \label{tab:task-api}
  \begin{tabular}{lll}
    \toprule
    メソッド & パス & 説明 \\
    \midrule
    GET  & \texttt{/}               & 課題・テストの一覧を表示する. \\
    GET  & \texttt{/tasks/:id}      & 指定 ID の課題・テストの詳細を表示する. \\
    GET  & \texttt{/create}         & 新規登録フォームを表示する. \\
    POST & \texttt{/create}         & フォーム入力から新しい課題・テストを作成し配列に追加する. \\
    GET  & \texttt{/edit/:id}       & 指定 ID の課題・テストの編集フォームを表示する. \\
    POST & \texttt{/edit/:id}       & 編集内容で既存の課題・テストを更新する. \\
    POST & \texttt{/delete/:id}     & 指定 ID の課題・テストを配列から削除する. \\
    \bottomrule
  \end{tabular}
\end{table}

一覧や詳細表示には HTTP GET を用い,
登録・更新・削除には HTTP POST を用いるという基本的な操作を統一している.

%%%%%%%%%%%%%%%%%%%%%%%%%%%%%%%%%%%%%%%%%%%%%%%%%%%%%%%%%%%%
\subsection{サブスク管理アプリのエンドポイント}

サブスク管理アプリ(\texttt{subs\_app})の主なエンドポイントを表\ref{tab:subs-api} に示す.

\begin{table}[h]
  \centering
  \caption{サブスク管理アプリのエンドポイント}
  \label{tab:subs-api}
  \begin{tabular}{lll}
    \toprule
    メソッド & パス & 説明 \\
    \midrule
    GET  & \texttt{/}              & サブスク一覧を表示する. \\
    GET  & \texttt{/subs/:id}      & 指定 ID のサブスク詳細を表示する. \\
    GET  & \texttt{/create}        & 新規登録フォームを表示する. \\
    POST & \texttt{/create}        & フォーム入力からサブスクを作成し配列に追加する. \\
    GET  & \texttt{/edit/:id}      & 指定 ID のサブスク編集フォームを表示する. \\
    POST & \texttt{/edit/:id}      & 編集内容で既存のサブスクを更新する. \\
    POST & \texttt{/delete/:id}    & 指定 ID のサブスクを配列から削除する. \\
    \bottomrule
  \end{tabular}
\end{table}

%%%%%%%%%%%%%%%%%%%%%%%%%%%%%%%%%%%%%%%%%%%%%%%%%%%%%%%%%%%%
\subsection{ToDo リストアプリのエンドポイント}

ToDo リストアプリ(\texttt{todos\_app})の主なエンドポイントを表\ref{tab:todos-api} に示す.

\begin{table}[h]
  \centering
  \caption{ToDo リストアプリのエンドポイント}
  \label{tab:todos-api}
  \begin{tabular}{lll}
    \toprule
    メソッド & パス & 説明 \\
    \midrule
    GET  & \texttt{/}              & ToDo の一覧を表示する. \\
    GET  & \texttt{/todos/:id}     & 指定 ID の ToDo 詳細を表示する. \\
    GET  & \texttt{/create}        & 新規登録フォームを表示する. \\
    POST & \texttt{/create}        & フォーム入力から ToDo を作成し配列に追加する. \\
    GET  & \texttt{/edit/:id}      & 指定 ID の ToDo 編集フォームを表示する. \\
    POST & \texttt{/edit/:id}      & 編集内容で既存の ToDo を更新する. \\
    POST & \texttt{/delete/:id}    & 指定 ID の ToDo を配列から削除する. \\
    \bottomrule
  \end{tabular}
\end{table}

%%%%%%%%%%%%%%%%%%%%%%%%%%%%%%%%%%%%%%%%%%%%%%%%%%%%%%%%%%%%
\subsection{動作環境と起動方法}

3 つのアプリケーションはいずれも,
Node.js がインストールされた環境で次の手順により起動する.
例として \texttt{tasks\_app} の場合を示す.

\begin{enumerate}
  \item リポジトリを取得し,\texttt{tasks\_app} ディレクトリに移動する.
  \item 一度だけ \verb|npm install| を実行して依存パッケージをインストールする.
  \item \verb|npm start| または \verb|node index.js| を実行し,
        \texttt{8080} 番ポートでサーバを起動する.
  \item ブラウザから \url{http://localhost:8080/} にアクセスすると,
        各アプリケーションのトップページが表示される.
\end{enumerate}

\noindent
サブスク管理アプリ,ToDo リストアプリについても,
それぞれ \texttt{subs\_app},\texttt{todos\_app} ディレクトリで
同様の手順を実行することで起動できる.

%%%%%%%%%%%%%%%%%%%%%%%%%%%%%%%%%%%%%%%%%%%%%%%%%%%%%%%%%%%%

\end{document}

